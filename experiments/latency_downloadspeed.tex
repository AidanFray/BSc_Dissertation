Figure~\ref{ref:LatencyDownload} shows the impact of latency on the connections throughput. The data was collated using an iperf server and client pair running over the host machine's localhost.

%Graph
\begin{center}
	\begin{tikzpicture}[ every axis plot/.append style={thick}]
		\begin{axis}[
			width=\linewidth,
			height=10cm,
			grid=major,
			xmin=10, xmax=100,
			ymin=0,
			xlabel=Latency (ms),
			ylabel=Throughput (Mbps)]
			\addplot+ [mark=none]
			table[search path=csv_data, col sep=comma]{LatencyDownload.csv};
		 \end{axis}
 	\end{tikzpicture}
\end{center}
\begin{figure}[h]
	\caption{Graph displaying Latency (ms) against Throughput (Mbps)}
	\label{ref:LatencyDownload}
\end{figure}

As it is evident, latency has an inverse relationship with that of the throughput. The reason for the reduction in throughput is due to the need to wait for acknowledgements. When TCP has sent a full window size (See~\ref{ref:window} Sliding Window) it will wait until acknowledgements have been received before sending more packets. This requirement to wait therefore means higher latency values increase the idle time of a TCP connection where extended periods of time are spent waiting for acknowledgements.

\clearpage
Inconsistency in the latency value also has an adverse effect on the throughput of the connection due to packets being deemed as ``lost" because they take longer than the time defined in the $RTO$ value. The $RTO$ variable defines the amount of time to wait before issuing a retransmission and deeming a packet as lost, this can be considered as packet loss where the TCP protocol will act accordingly. RFC 6298 \citep{paxson2011computing} defines the calculation of $RTO$ to involve the two state variables: $RTTVAR$ (Round-trip time variation) and $SRTT$ (Smoothed round-trip time). These values are updated each time there is a new RTT measurement.

\begin{center}
\[RTO = SRTT + MAX(G, K * RTTVAR)\]
\end{center}
where: \\
\hspace*{1cm} $K$ is 4 \\
\hspace*{1cm} $G$ is Time Granularity

\begin{figure}[h]
	\caption{Equation for calculating $RTO$}
	\label{ref:RTO} 
\end{figure}

Figure~\ref{ref:RTO} contains the main equation for calculating the $RTO$ value. This equation is used to take into the residual RTT of a network when calculating the expected retransmission time. This means large networks with large latency values are taken into account.
