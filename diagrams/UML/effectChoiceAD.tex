\newcommand{\widthEffectChoice}{4cm}

\tikzset{%
	>={Latex[width=2mm,length=2mm]},
	% Specifications for style of nodes:
            base/.style = {draw=black,
                           minimum width=\widthEffectChoice, 
                           text width=\widthEffectChoice, 
                           minimum height=1cm, 
                           text centered},                           
            decision/.style = {base, diamond, fill=yellow!15},
            normal/.style = {base, rectangle, rounded corners}}

\begin{tikzpicture}

	\node (start) [shape=circle, fill=black]{};
	
	\node (setQueue) [right of=start, normal, xshift=\widthEffectChoice]
	{Binds the NFQUEUE's `Mode' to chosen effect};
	
	\node (setVariables) [right of=setQueue, normal, xshift=\widthEffectChoice]
	{Sets all \\ preference \\ variables by parsed \\ arguments};
	
	\node (userOutput) [fill=blue!15, below of=setVariables, normal, yshift=-\widthEffectChoice + 2cm]
	{Displays summary of chosen modes and prefrences to the user};
	
	\node (runNfqueue) [left of=userOutput, normal, xshift=-\widthEffectChoice]
	{NFQUEUE Object is then run. That will pass all packets to the assigned 'Mode'};

	\node (end) [shape=circle, fill=black, below of=start, yshift=-\widthEffectChoice + 2cm]{};

	%Arrows
	\draw[->] (start)--(setQueue);
	\draw[->] (setQueue)--(setVariables);
	\draw[->] (setVariables)--(userOutput);
	\draw[->] (userOutput)--(runNfqueue);
	\draw[->] (runNfqueue)--(end);

\end{tikzpicture}