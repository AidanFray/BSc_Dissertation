\small
\begin{longtable}{| p{1.5cm} | p{1.5cm} | p{4cm} | p{5cm} | p{3cm} |}
	\hline
		%HEADER
		\bf{Section} & \bf{Test Code} & \bf{Test Name} & \bf{Description} & \bf{Expected} \\ \hline
		
		%--- UDP CLIENT
		\bf{UDP Client}&&&& \\ \hline
		
		% Business Logic
		Logic &&&& \\ \hline
		&UC-L1& TestStartUp() & Creates a client object before each test & n/a\\ \hline
		&UC-L2& TestCleanUp() & No clean-up needed & n/a \\ \hline
		&UC-L3& Connect-Disconnect() & The client connects performs no action then disconnects & n/a \\ \hline
		&UC-L4& CreateClientObj() & A client object is created & n/a \\ \hline
		&UC-L5& SendPacket() & Sends a single UDP packet & n/a\\ \hline
		&UC-L6& CheckGridSend() & Checks the method sends out the correct number this depends on grid size & n/a \\ \hline
		
		% UI
		UI &&&& \\ \hline
		&UC-U1& TestStartUp() & Loads up a fresh window from the command line & n/a \\ \hline
		&UC-U2& TestCleanUp() & Clicks the X in the top right corner & n/a \\ \hline
		&UC-U3& ConnectClick() & Clicks the "Connect" button & n/a \\ \hline
		&UC-U4& RestartClick() & Clicks the "Connect" button then the "Restart" button & n/a \\ \hline
		&UC-U5& Connect-InvertCheck() & Clicks the "Connect" button and checks all the buttons invert their "Enabled" property correctly & Connect(Disabled), Restart(Enabled) \\\hline
		&UC-U6& Restart-InvertCheck() & Clicks the "Connect" button, then the "Reset" button and checks the "Enabled" properties for all buttons are correct & Connect(Enabled), Restart(Disabled) \\\hline
		&UC-U7& TextEntry() & Enters text into the text box and then "Connect" is clicked & n/a \\ \hline
				
		%--- UDP SERVER
		\bf{UDP Server} &&&& \\ \hline		
		
		%Business 
		Logic &&&& \\ \hline
		&US-L1& TestStartUp() & Creates the client object & n/a \\ \hline
		&US-L2& TestCleanUp() & n/a & n/a  \\ \hline
		&US-L3& CreateObject() & Tests that the server object can be created successfully & n/a \\ \hline
		&US-L4& WaitForTimeout() & Starts the server and checks it closes cleanly after time-out & n/a \\ \hline
		
		% UI
		UI &&&& \\ \hline
		&US-U1& TestStartUp() & 
		Loads up a fresh window from the CMD line & n/a \\ \hline

		&US-U2& TestCleanUp() & 
		Clicks the X in the top right hand corner & n/a  \\ \hline
		
		&US-U3& StartClick() & Click the "Start" button & n/a \\ \hline
		&US-U4& RandomiseClick() & Click the "Randomise" button & n/a \\ \hline
		&US-U5& Start-ResetClick() & Clicks the "Start" button then resets and checks it returns to it's default state & n/a \\ \hline
		&US-U6& Randomise-ResetClick() & Clicks the "Randomise" button, the resets and checks it returns to it's default state & n/a \\ \hline
		&US-U7& Start-CheckInvert() & "Start" button clicked and then the test checks for "Enabled" property is inverted & n/a \\ \hline
		&US-U8& RestartCheckInvert() & "Start" then "Restart" button clicked and the test then checks for the correct inverting of the buttons "Enabled" property Start = Enabled, Restart = Disabled & n/a \\ \hline
		&US-U9& Stat-PacketLossReset() & Checks when the restart button is clicked the statistics is returned to default & "-" \\ \hline
		&US-U10& Stat-TotalPacketLossReset() & Checks when the restart button is clicked the statistics is returned to default & "-" \\ \hline
		&US-U11& Stat-TotalPacketSentReset() & Checks when the restart button is clicked the statistics is returned to default & "-" \\ \hline
		&US-U12& Stat-CheckDeafult() & "Start" is clicked and the test then waits for a time-out, it then checks the default values of the statistics & PacketLoss(100) \\ \hline
		
		%--- UDP COMBINED		
		Live Tests &&&& \\ \hline
		&U-1& TestStartUp() & Both windows were opened on the command line & n/a \\ \hline
		&U-2& TestCleanUp() & Both windows are closed by automated actions & n/a \\ \hline
		&U-3& SendAndRecieve-Valid() & The client sends packages to the server and the server accepts packets & n/a \\ \hline
		&U-4& SendAndRecive-Valid-Twice() & The performs the same action at the above test but does the whole loop an extra time to check if the reset works correctly & n/a \\ \hline
		&U-5& SendAndRecieve-Invalid() & This test doesn't connect the client to the server but a random IP on the network, this is to check the server will act accordingly & n/a \\ \hline
\end{longtable}

Packet Script Test Plan for the default effects.

\begin{longtable}{| p{2cm} | p{0.5cm} | p{4cm} | p{5cm} | p{3cm} |}
	\hline	
	\bf{Section} & \bf{Test No} & \bf{Test Name} & \bf{Description} & \bf{Expected} \\ \hline
	\bf{Effects} &&&& \\ \hline
	
	% -- 	
	PacketLoss &&&& \\ \hline
	&P-1	& StartPacketLoss() 
	& Starts the script with the expected parameter (e.g. -pl 10) &  Script should start PacketLoss mode \\ \hline
	
	&P-1	& PacketLossEffect() 
	& Pings the script and checks the effect on the ping packets, the test pings until one packet is lost. Checking for exact percentages isn't reliable enough & Some packet loss \\ \hline	
	
	% -- 	
	Latency &&&& \\ \hline
	&L-1& StartLatency() 	& Starts the script with valid parameters for latency (e.g. -l 10) & Script should start in Latency mode \\ \hline
	&L-2& LatencyEffect() 	& Pings the script and checks the latency value within a margin of error & Latency effect 		on ping packets \\ \hline
	
	% -- 
	Bandwidth &&&& \\\hline
	&B-1& StartBandwidth() &  Starts the script with the expected parameters (e.g. -rl 100) & Script should start in Bandwidth limit mode \\ \hline
	&B-2& BandwidthEffect() & Starts bandwidth mode and starts pinging the localhost, after a certain amount of packets the rate is calculated and checked against the rate limit value & n/a \\\hline
	
	Out-Of-Order &&&& \\\hline
	&O-1& StartOrder() & Starts the script and checks the effect initialises correctly & n/a \\\hline
	&O-2& OutOfOrderEffect() & Starts the effect and checks using the localhost that the packet sequence numbers are not in sequential order when they return & n/a \\\hline
	
	Jitter &&&& \\\hline
	&J-1& StartJitter() & Starts the script in the mode and checks all initialisation finishes correctly & n/a \\\hline
	&J-2& JitterEffect() & Ping packets are sent on the localhost and a difference in packet latency is looked for  & n/a\\\hline
	
	% -- 	
	Validation &&&& \\ \hline
	&V-1& LatencyValidation() & Check that the validation works for the latency mode & Range = 1-1000ms \\\hline
	&V-2& PacketLossValidation() & Check that the validation works for the packet loss mode & Range = 1-100\%\\\hline
	&V-3& BandwidthValidation() & Checks that validation works for bandwidth mode & Range = 1-10000B/s\\\hline
	&V-4& JitterValidation & Checks that validation works for difference values & Range = 10ms-100ms \\\hline	
	
\end{longtable}