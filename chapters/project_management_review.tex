\chapter{Project Management Review}

\section{Project progress}
Appendix C contains the original time plan from the initial report. So far the project is on time for the majority of deadlines. The degradation script is able to cause degradation on a connection with around 6 different effects, these effects are all controllable by a GUI. The server and client coding is complete with a relatively comprehensive set of unit tests to accompany it. The time allocation of 6 weeks for development of the script in hindsight was too short and will require an extension, to remain on track further development of the script and testing will be done concurrently for the next month.

\section{Changes from initial design}
There have however been some changes to the initial design of the client server pairs (this has been touched on previously, in the Designs section). HTTP has decided not to have a custom solution created for it, instead the network will have a client attempting to connect to a web page through various modern browsers, this was decided because of its visual and real world application where degradation will more visible and understood easier than the HTTP client and server pair. 
Another change from the initial design was the removal of the of the FTP client where FileZilla took its place, this was because of the complexity of a working FTP client, it would require extensive time invested to create a fully functioning client where this time invested would provide nothing to the initial goal, this therefore meant that the open source FileZilla would be a perfect solution to this problem.

%\section{Project management techniques}
%Could talk here about using GitHub Issues?

\section{Issues and obstacles}
As stated in the introduction there were various issues that have been negated or overcome. The first issue was the max capacity of the degradation tool, it was initially able to handle small amounts of traffic, this was deemed as too low, this was solved by introducing concurrency into the script, a pool of threads are created at run-time and once a packet is received it allocates the job to a thread in the pool, this resulted in minimal work being done by the main thread. The result was a considerable increase in the speed of the max transfer rate. This could be improved further with more efficient languages and techniques. 

\section{Risks and ethics}
There are various risks for this project, the first being the aspect of not completing all the objectives of the project, because of the current progress of the project this risk has a low possibility of occurring, if this risk became a reality the solution would be to discuss which aspects were missed and why and how they affect the project. Another common risk for software development projects similar to this is data loss or loss of progress, this is however not an issue due to the project fully using a version control system where commits are being submitted after small changes and reordered, this therefore negates the issue of any chance of data loss.

Ethics of this project is also an aspect to consider. The tool has the potential to cause negative effects on a live network, this therefore means that the tool is not to be released after this project to not allow it to be abused. This also means the tool because of ethical concerns cannot be used on a network without the required permissions.


%===========================================================================================================================%
%In this section you are required to provide an overview of your project management to date.  The first question is how well your progress matches up to your initial time plan, and what adjustments you need to make if any.  Then you must provide a discussion of potential risks for the project and product, and finally address any ethical concerns and your responses.}
%
%Project Progress:
%Referring to the task list and time plan from your initial report (reproduced as an appendix), you should briefly discuss how far you have progressed, where your progress varies from your original time plan estimate, and what revisions to your time plan you may intend to make for the remaining weeks. If there are significant variations, a fresh time plan should be presented.

