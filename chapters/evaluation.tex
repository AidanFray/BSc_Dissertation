%In this chapter you must evaluate your overall project achievement, and provide a clear statement of where the delivered results stand in relation to the original project goals. You should also provide an outline of any further potential work which you can foresee as beneficial to the future development of your project.
\chapter{Evaluation}


%The tool is used to create results and there're discussed here
\section{Experimentation Results}

	\subsection{Effect of Packet Loss on Download Speeds}
	The graph above shows the download speed measured in Mbps against the increasing rate of packet loss, as you can see the download rate only reaches around 25\%, this is due to the connection test failing at anything above. The tests were performed by using Iperf on the localhost as to reduce any degradation that may exists already in an existing network.

%Graph
\begin{center}
	\begin{tikzpicture}[ every axis plot/.append style={ultra thick}]
		\begin{axis}[
			title=\underline{Packet Loss X Download Rate},
			width=\linewidth,
			height=10cm,
			grid=major,
			xmin=0, xmax=25,
			ymin=0,
			xlabel=Packet Loss (\%),
			ylabel=Download Speed (Mbps)]
			\addplot table [mark=none, search path=csv_data, col sep=comma]{PacketLossDownload.csv};
		 \end{axis}
 	\end{tikzpicture}
\end{center}

My initial hypothesis for the sharp decreases in packet loss from 0\% to 1\% is due to the congestion control mechanisms built into TCP. TCP as mentioned in the background section allows for multiple devices to co-exist on a network and needs to dynamically balance network resources for every client, it performs this through congestion control algorithms where packet loss in this aspect is assumed as a sign of congestion to the algorithm where it adapts by reducing its transfer rate each time a packet is lost, this means the transfer rate drastically reduces until it reaches a platform. The transfer rate is dictated by a few factors, one known as the congestion window (Cwnd) and the advertised size of the receiving window (Rwnd). The overall window size is the smaller of these two numbers, and the smaller the window size, the less packets can be send in one time frame. The congestion window therefore is reduces every time packet loss is detected. 

The test devised to prove this hypothesis will involve once again an iperf client/server where a packet is dropped after a time period while the congestion window size is tracked. Below is a graph showing the result of a test spanning 10 seconds where a single incoming packet was dropped every 1 second, the graph is measuring Cwnd sizes on the y axis and Time on the x axis.

%Graph
\begin{center}
	\begin{tikzpicture}[ every axis plot/.append style={ultra thick}]
		\begin{axis}[
			title=\underline{Cwnd X Time},
			mark=square,
			width=\linewidth,
			height=10cm, 
			ymin=0, ymax=3400,
			xmin=0, xmax=9.9,
			grid=major,
			xlabel=Time (Seconds),
			ylabel=Cwnd]
			\addplot table [mark=none, search path=csv_data, col sep=comma]{PacketLossCwnd.csv};
		 \end{axis}
 	\end{tikzpicture}
\end{center}

As you can see around the 1 second mark there are almost identical drops in the Cwnd size, this demonstrates that packet loss causes the congestion window to reduce in size and if packet loss is substantial enough can cause a huge drop in transfer rates.

This is known in live networks as the `sawtooth' pattern where the client is probing for more bandwidth.

%Graph
\begin{center}
	\begin{tikzpicture}[every axis plot/.append style={ultra thick}]
		\begin{axis}[
			title=\underline{Cwnd X Time},
			mark=square,
			width=\linewidth,
			height=10cm, 
			ymin=0, 
			xmin=0, xmax=10,
			grid=major,
			xlabel=Time (Seconds),
			ylabel=Cwnd]
			\addplot table [mark=none, search path=csv_data, col sep=comma]{PacketLossLiveNetworkCwnd.csv};
		 \end{axis}
 	\end{tikzpicture}
\end{center}

Please also note the initial increase in transfer rate, this is displayed in the graph below:

%Graph
\begin{center}
	\begin{tikzpicture}[every axis plot/.append style={ultra thick}]
		\begin{axis}[
			title=\underline{Cwnd X Time},
			mark=square,
			width=\linewidth,
			height=10cm, 
			ymin=0, 
			xmin=0,
			grid=major,
			xlabel=Time (Seconds),
			ylabel=Cwnd]
			\addplot table [mark=none, search path=csv_data, col sep=comma]{PacketLossCwnd_SlowStart.csv};
		 \end{axis}
 	\end{tikzpicture}
\end{center}


This is known as the TCP Slow start and this is used by the protocol to naturally align itself into the network eco-system, it starts off with a modest transfer rate to prevent situations where an initial abnormally large transfer rate would fill buffers and cause network congestion and adjusts by incorporating the techniques explained above to manually adjust its transfer rate to fit naturally into the current network.
 

	%TODO - Comparason between congestion algos 
	%			- Reno
	%			- Cubic

%Project Achievements:
%You should consider your original goals and statewh clearly ich have or have not been fully or partially met, and discuss reasons for any significant variance.
\clearpage
\section{Project Achievements}

{\bf Initial Goal}\\
{\it Create a custom simulated network that can demonstrate and visualise network degradation and common DoS attacks, this is so network engineers can identify weak spots and points of strain}

% URLS
\newcommand{\chromeUrl}{\url{https://www.google.com/chrome/index.html}}
\newcommand{\firefoxUrl}{\url{https://www.mozilla.org/en-US/firefox/new/}}
\newcommand{\fileZillaUrl}{\url{https://filezilla-project.org/}}

\subsection{Objective One}
{\it Develop a fully working clients and servers that communicate using HTTP/1.1 and FTP over the TCP protocol alongside a client and server that use the UDP protocol}

This objective initially involved the creation of custom client server solution for HTTP, FTP and UDP. This involved a lengthy development period where each client and server would require the implementation of required features for the project. It was decided after a period of 3.5 weeks that the simulated network having a custom solution created no added value to the initial goal of the project, this is why HTTP has been replaced by pre-existing browsers such as Google Chrome \footnote{\chromeUrl} or Mozilla Firefox \footnote{\firefoxUrl} and and FTP solution replaced by FileZilla's \footnote{\fileZillaUrl} server and client pair. These are all projects that have had multiple years invested into their development and offer increased stability and selection of features with next to zero investment in time development time for this project and they have all the features required to meet the goal of this project. UDP on the other hand was left as a custom solution, for the sacrifice of realism the UDP client and server show a easy to view visualisation of UDP packet loss and is very effective at what it does.

However, even with the changes each sub point of the object has been met.

\subsection{Objective Two}
{\it Create a program that runs on a Linux based OS that can be used to simulate degradation and attacks. This program will then be run on the router.}

In the aims and objectives section lists out the criteria required to meet this objective. The created solution meets all of these criteria. It provides the functionality to simulate packet loss, latency and bandwidth plus a selection of other effects. The other effects were chosen to provide more depth to the project and allow the possibility to chain effects together to simulate more sophisticated degradation. The script also allows UDP flooding and ARP spamming. It contains a mode to increment the TTL but this mode was discarded from the main advertised functionality due to its inherit difficulty in displaying its effect on a connection due to the hard task for forming loops in a networks structure. This script is also functional on the set-up Raspberry Pi Router.


\subsection{Objective Three}
{\it Create a working custom router that can be used to simulate degradation in conjunction with the program outlined in Objective 2}

The project has met the requirements of this objective. The router can be used in place of a commercial router and can handle around 3 clients at a single time while still providing acceptable transfer speeds. A client can connect to the router by wireless or wired means.

\subsection{Overall}
%TODO - Explain how the overall goal has been met

%Further work:
%You should identify any specific areas of the existing product where further work is required, either to address known bugs or omissions, or to improve the system – for example an area you developed at an early stage may in hindsight have been structured in a better way, although it is still perfectly functional.
%You should also provide an outline of any additional areas of potential development you can envisage beyond the initial goals, where the system might be extended in future for greater functionality.
\section{Further work}
%TODO